\documentclass[12pt]{article}

\usepackage[margin=1in]{geometry} 
\usepackage{graphicx} 
\usepackage{amsmath,amsfonts,amsthm} 
\usepackage{hyperref}

\newtheorem{thm}{Theorem}[section]
\newtheorem{lem}[thm]{Lemma}
\newtheorem{prop}[thm]{Proposition}
\newtheorem{cor}[thm]{Corollary}
\newtheorem{conj}[thm]{Conjecture}


\begin{document}
\title{\bf A Tropical Approach to Fair Division}
\author{Josephine Yu and Charles Wang}
\date{\today}

\maketitle

\begin{abstract}
-
\end{abstract}


\section{Introduction}

The study of this problem is motivated by \href{https://www.math.hmc.edu/~su/papers.dir/rent.pdf}{[Su99]} and \href{http://wwwhomes.uni-bielefeld.de/imw-wp/files//imw-wp-311.pdf}{[HRS01]}. We focus on the rental harmony problem presented in [Su99] (quite potentially generalizable, though). [HRS01] presents an algorithm for computing the solution. However, neither paper presents a tropical interpretation for the problem, which in some sense may be the most natural.

\subsection{Problem Statement}

$n$ people want to rent $n$ rooms for a total rent of $R$ dollars. Each person, $p_i$, is willing to pay at most $v_{ij}$ for room $j$. This defines a value matrix, $V\in\mathbb{R}_{\ge0}$: 
\[V= \left( \begin{array}{rrrr}
 v_{11} & v_{12} & \cdots & v_{1n} \\
 v_{21} & v_{22} & \cdots & v_{2n} \\
 \vdots & \vdots & \ddots & \vdots \\
 v_{m1} & v_{m2} & \cdots & v_{mn}
 \end{array} \right)\]
where $v_{i1}+v_{i2}+...+v_{in}=R$ for  $1\le i\le n$.
\\\\
An assignment is a permutation on $\{1,2,...,n\}$ that assigns to each person $i$ a unique room $j$ and a price $(p_1,p_2,...,p_n)$ is a distribution of the total rent among the $n$ rooms. The value of an assignment is the sum of the values for each person-room  pair in the assignment. The utility of person $i$ renting room $j$ for price $p_j$ is $v_{ij}-p_j$. 
\\\\
 The goal is to find an assignment such that no person pays more than what they are willing to for their assigned room (nonnegative utility). As we will see, this question itself is quite simple. However, there are interesting questions that can be asked. For example, even though each person gets nonnegative utility for each room, they may want another room more than their current room given the pricing. In this case, we have an \textit{envious} assignment. A natural question to ask is whether there is an envy-free assignment. Another natural question to ask is whether there is a fairest assignment (perhaps one which gives each person the same utility). 
\section{Rental  Harmony}

Given the basic problem, an easy solution is clear. We use this section to introduce the tropical interpretation of the problem. 

\begin{lem}
If the value of an assignment is at least the total rent, then there is a price for that assignment such that each person has nonnegative utility.
\end{lem}

\begin{proof}
Let $v$ denote the value of the assignment and $v_i$ denote the value person $i$ receives from her room in the assignment. Let the price be $(\frac{Rv_1}{v},\frac{Rv_2}{v},...,\frac{Rv_3}{v})$. Then since $v\ge R\implies \frac{R}{v}\le1$, we have $v_i\ge \frac{Rv_i}{v}$ for all $i$. Finally, $\sum_i \frac{Rv_i}{v}=\frac{R(v_1+v_2+...+v_n)}{v}=\frac{Rv}{v}=R$.
\end{proof}

\begin{lem}
For any value matrix, $V$, there is always an assignment whose value is at least the total rent. 
\end{lem}

\begin{proof}
Suppose not. Then every assignment must have value less than $R$. There are $n!$ assignments corresponding to the permutations on $n$ elements. Consider the sum of these assignments. Notice that each $v_{ij}$ appears in the sum $(n-1)!$ times by fixing one element at a time. Thus: \[(n-1)!\sum_{i,j} v_{ij}< n!R\implies\sum_{i,j}v_{ij}<nR.\]
But since, for each $i$, $v_{i1}+v_{i2}+...+v_{in}=R$, we must have that $\sum_{i,j}v_{ij}=nR$. Thus, we have a contradiction so there is always some assignment whose value is at least $R$.
\end{proof}
\noindent
In particular, the maximum is exactly the tropical determinant and by the above lemma is guaranteed to have value at least $R$. (If it is equal to $R$, then $v_{ij}=\frac{R}{n}$ for all $i,j$ and the problem is symmetric so we will assume generally that this value is greater than $R$.)

\begin{thm}
There is always an assignment such that each person has nonnegative utility.
\end{thm}

\begin{proof}
This is a straightforward application of the previous two lemmas. Furthermore, Lemma 2.1 provides a construction for this price.
\end{proof}

%Example where this price assignment goes wrong (has envy) and is not fair
%Need for "better" price

\noindent

\bibliographystyle{plain}

\bibliography{paper}


\end{document}